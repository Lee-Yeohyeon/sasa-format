\documentclass[9pt,showtrims,twoside,openright,chapter]{oblivoir}

\usepackage[stock]{fapapersize}

\usefapapersize{128mm,192mm,22.5mm,*,20mm,15mm}

%% 색깔 설정

\usepackage[dvipsnames,svgnames,x11names]{xcolor}
% Pantone Fashion Color: Spring 2014
\definecolor{PlacidBlue}{cmyk}{.47,.17,.02,.0}
\definecolor{VioletTulip}{cmyk}{.44,.39,.0,.0}
\definecolor{Hemlock}{cmyk}{.39,.04,.35,.0}
\definecolor{Paloma}{cmyk}{.35,.24,.27,.0}
\definecolor{Sand}{cmyk}{.20,.27,.48,.0}
\definecolor{Freesia}{cmyk}{.0,.14,.100,.0}
\definecolor{Cayenne}{cmyk}{.06,.74,.56,.0}
\definecolor{CelosiaOrange}{cmyk}{.0,.63,.80,.0}
\definecolor{RadiantOrchid}{cmyk}{.32,.65,.0,.0}
\definecolor{DazzlingBlue}{cmyk}{.92,.57,.0,.0}
\definecolor{PurpleHaze}{cmyk}{.56,.51,.15,.0}
\definecolor{Comfrey}{cmyk}{.74,.28,.63,.10}
\definecolor{MagentaPurple}{cmyk}{.51,.94,.24,.24}

\colorlet{MainColorOne}{Comfrey}
\colorlet{MainColorTwo}{PurpleHaze}

\usepackage{tikz}
\usepackage{calc}

\graphicspath{{figure/}}  


\usepackage{tabu}

\makepagestyle{KNUworkshop}

\makeheadrule{KNUworkshop}{0pt}{0pt}

% 쪽 번호 자릿수. 세 자리 숫자가 들어갈 폭을 가지도록 함. 
\newlength\pagenumwidth
\settowidth{\pagenumwidth}{999}

%% 쪽 번호 들어가는 상자
\tikzset{pagefooter/.style={
anchor=base,font=\sffamily\bfseries\small,
text=white,fill=MainColorOne!60,text centered,
text depth=20mm,text width=\pagenumwidth}}

\makeoddhead{KNUworkshop}{}{}{%
\sffamily\footnotesize%
\begin{tikzpicture}[xshift=-.75\baselineskip,yshift=.25\baselineskip,overlay,fill=MainColorOne!50,draw=MainColorOne!50]
\fill circle(3pt);
\draw[semithick](0,0) -- (-\textwidth-2\spinemargin,0);
\end{tikzpicture} \rightmark}

\makeevenhead{KNUworkshop}{%
\sffamily\footnotesize\leftmark\,\, 
\begin{tikzpicture}[xshift=.5\baselineskip,yshift=.25\baselineskip,overlay,fill=MainColorOne!50,draw=MainColorOne!50]\fill (0,0) circle (3pt); \draw[semithick](\textwidth+2\spinemargin,0) -- (0,0);\end{tikzpicture}}{}{}

\makeoddfoot{KNUworkshop}{}{}{\tikz[baseline]\node[pagefooter]{\thepage};}

\makeevenfoot{KNUworkshop}{\tikz[baseline]\node[pagefooter]{\thepage};}{}{}

\copypagestyle{chapter}{KNUworkshop}
\makeoddhead{chapter}{}{}{}
\makeevenhead{chapter}{}{}{}


\makeatletter
\def\@hgpsmarks{%
      \let\@mkboth\markboth
      \def\chaptermark##1{%
        \markboth{%
          \ifnum \c@secnumdepth >\m@ne
            \if@mainmatter
              \hchaptertitlehead\enskip%\ %
            \fi
          \fi
          ##1}{}}%
      \def\tocmark{\markboth{\contentsname}{\contentsname}}%
      \def\lofmark{\markboth{\listfigurename}{\listfigurename}}%
      \def\lotmark{\markboth{\listtablename}{\listtablename}}%
      \def\bibmark{\markboth{\bibname}{\bibname}}%
      \def\indexmark{\markboth{\indexname}{\indexname}}%
      \def\sectionmark##1{%
        \markright{%
%% disabled printing \thesection.
%          \ifnum \c@secnumdepth >\z@
%          \fi
% enabled printing \thesection.
          \ifnum \c@secnumdepth >\z@
            \thesection \enskip\ %
          \fi
          ##1}}%
    }
\makepsmarks{KNUworkshop}{\@hgpsmarks}


%장제목 스타일
\makechapterstyle{KNUchapter}{%
   \setlength{\afterchapskip}{40pt}
  \renewcommand*{\chapterheadstart}{\vspace*{-3\onelineskip}}
  \renewcommand*{\afterchapternum}{\par\nobreak\vskip 25pt}
   \renewcommand*{\chapnamefont}{\normalfont\LARGE\flushright}
   \renewcommand*{\chapnumfont}{\normalfont\HUGE\color{MainColorOne}}
   \renewcommand*{\chaptitlefont}{\color{MainColorOne}\sffamily\HUGE\bfseries\flushright}%\flushright}
%   \renewcommand*{\printchaptername}{%
%     \chapnamefont\MakeUppercase{\@chapapp}}
  \renewcommand*{\prechapternum}{\chapnamefont\MakeUppercase{Chapter}}
  \renewcommand*{\postchapternum}{%
			 \makebox[0pt][l]{%
	        \hspace{1em}{\color{MainColorOne!50}\rule{\midchapskip+\spinemargin}{\beforechapskip}}
			}}
   \renewcommand*{\chapternamenum}{}
  \setlength{\beforechapskip}{18mm}%  \numberheight
  \setlength{\midchapskip}{\paperwidth}% \barlength
  \addtolength{\midchapskip}{-\textwidth}
  \addtolength{\midchapskip}{-\spinemargin}
   \renewcommand*{\printchapternum}{%
       \hspace{1em}\resizebox{!}{\beforechapskip}{\chapnumfont \thechapter}}
   \makeoddfoot{plain}{}{}{\normalfont\normalsize\sffamily\thepage}
}

\makeatother

\usepackage{ob-chapstyles}


\usepackage{makeidx}
\DeclareRobustCommand{\myem}[2][\empty]{%
\ifx#1\empty
#2\index{#2}%
\else
#2\index{#2}(#1)\index{#1}%
\fi}


\makeindex

\usepackage{amsmath,amsthm}

\newtheorem{thm}{정리}[chapter]
\newtheorem{cor}[thm]{따름정리}
\renewcommand{\proofname}{증명}
% 줄간격
\linespread{1.4}
\everydisplay{\setstretch{1.2}}

\title{수학 에세이 과제}
\author{1208 이승민}

\begin{document}
\maketitle
\mainmatter
\pagestyle{KNUworkshop}
\chapterstyle{KNUchapter}

\chapter{2019}
\indent
\hspace{10pt} 다비드는 별로 기분이 좋지 않았다. 그의 집 앞 화원은 언제나 특유의 정갈한 패턴을 유지하고 있었지만, 오늘은 아니었다. 정체모를 이물질이 화원에 침범해 있었다. 그는 어떤 패턴이 흐트러지는 것을 매우 마음에 들어 하지 않는 사람이었다. 지난 60년간 패턴이 바뀌는 일은 있었지만 결코 패턴이 깨지는 일은 없었다. 중대한 사태가 일어난 것이다. 그는 늘상 그랬듯 상황을 논리적으로 분석했으나, 정보와 근거가 부족했으므로 결론을 신뢰할 수 없는 상태였다. 때문에 다비드는 이물질을 조사해보기로 결심했다. \par
 다비드는 이물질을 집어들었다. 이물질은 파피루스였다. 파피루스에는 무엇인가 기록되어 있었는데, 영어와 숫자 및 수학 기호였으므로 다비드도 읽을 수 있었다. 다비드는 첫 글자부터 마지막 글자까지 멈추지 않고 읽어내려갔다. \par
 “......” \par
 전부 읽은 후에도 그는 파피루스를 계속해서 응시했다. 겉으로는 드러나지 않았지만, 그의 눈동자 깊은 곳에서 전등이 켜진 듯 빛이 뿜어져 나오고 있었다. 파피루스 안에는 그가 평생 바라오던 것이 들어 있었으므로.
 \\

 다비드의 조부는 ‘완전한 수학 체계’를 만들기 위해 평생을 바쳤다. 그의 아버지 또한 그랬고, 그도 마찬가지였다. 그것은 일종의 숙원이었다. 다비드가 찾고 있는 것은 무모순인 동시에 완전한 공리계였다. 그의 생각에는, 이것이 발견된다면 수학은 완성되는 것이었다. 모든 명제들이 그 체계 안에서 증명가능할 것이다. 얼마나 아름다운 일인가? 인류는 끝없이 질문을 던지며 동시에 그 질문에 답할 것이다. 수학은 더욱 발전하여 새로운 수학 도구들이 만들어질 것이며, 또한 그것이 수학의 발전을 가속시킬 것이다. 새로운 개념과 정리들은 온갖 과학기술에 응용되어 인류의 삶을 더욱 윤택하게 만들 것이다. 지식과 기술의 극한을 체험할 수 있다. 
 그것에 더해, ‘완성’이 가능할 것이다. 공리계 내부에서 모든 명제들이 증명가능하다는 말은, 모든 명제를 발견하고 증명한다면 수학이 완결된다는 것을 의미한다. 아직 증명되지 않았을 뿐 모든 것들이 증명될 수 있는 것이다. 현재의 수학 체계로는 불가능한 일이다. 얼마나 긴 시간이 걸릴지 알 수 없지만, 완전한 수학 체계가 있다면 먼 미래에는 실현될 수 있다. 
 \\ \par
 다비드는 떨리는 손으로 읽고 있던 조부의 논문을 내려놓았다. 탁한 담배 연기가 지하실을 떠돌았다. 그의 예상이 사실로 확인되었다. 파피루스에 기록된 것은 ‘완전하고 무모순인 공리계’였다. 아무리 찾아보아도 그 어떤 오류도 보이지 않았다. 그의 뇌리에서 수많은 상념들이 소용돌이쳤다. 이 파피루스는 어디서 온 것일까. 왜 하필 나에게 왔나. 조부님이 이걸 보셨다면 무슨 말을 하셨을까. \par
 “아니, 그런 것은 중요하지 않지.” \par
 다비드는 말했다. 자신이 생각해낸 것은 아니었지만, 이것이 학계에 어떤 반응을 불러일으킬지 상상만 해도 즐거웠다. 그는 틀리지 않았다. \\
다비드는 일어섰다. 일어서서 지하실을 걸어나갔다.
\\
\par
\chapter{2137}

 “오늘은 ‘완전공리계’에 대해 배워보죠. 21세기의 위대한 수학자 다비드 데이비드는 완전공리계를 발표한 뒤로 일약 세계적인 스타가 되었습니다. 수학의 체계를 세운 이론이었으니까요. 그러나 당시 수학자들의 반응은 좋지 않았습니다. 반대학파의 수장 쿠르트는 인정할 수 없다며 격렬히 반박했고, 많은 수학자들이 완전공리계의 모순을 찾기 위해 애를 썼습니다. 하지만 당연하게도....”
 \\

 맥은 별로 기분이 좋지 않았다. 수학 과목에서 F를 받았기 때문이었다. 재수강을 해야 한다고 생각하니 속이 메슥거렸다. \par
 “젠장. 아버지께 뭐라고 변명하지?” \par
 “무릎꿇고 빌어.” \par
 맥은 갑자기 성적표를 위조할 수 있었던 시대가 그리워졌다. 오늘 들은 다비드의 시대에는 성적표 조작이 가능했었다고 한다. 다비드야 성적을 조작할 필요가 없었겠지만. \par
 “다비드는 수학을 잘했겠지. 부럽네.” \par
 “야, 택시 온다.”
\\ \par
 맥은 조용히 집안으로 들어왔다. 예상대로 아버지가 있었다. 맥은 선수를 치기로 마음먹었다. \par
 “아버지. 대체 수학은 왜 배우는 건가요?” \par
 정적이 흘렀다. 아버지는 야릇한 미소를 지었다. 맥은 짙은 불안감에 휩싸였다. \par
 “이제는 배울 필요가 없다.” \par
 “죄송합...예?” \par
 “말 그대로다. 배울 필요가 없다.” \par
 “.....” \par
 맥은 아버지의 말을 들었지만, 이해하지는 못했다. 아버지는 설명을 계속했다. \par
 “오늘이다. 오늘밤에 EM의 가동이 시작된다. 너는 모르고 있었겠지만, 여태까지 나는 EM이란 수학 인공지능 프로그램을 만들어왔다. EM의 기능은 공리계 내부의 모든 수학적 개념, 정리, 성질들을 발견하고 증명하는 것이다. EM은 여태까지 있었던 수학 정리들 사이의 모든 상관관계를 분석하여 새로운 정리를 알아낸다. 지금 현대 수학이 하고 있는 일을 완벽히 해낼 수 있는 것이다.” \par
 맥은 아버지가 지금 굉장히 진지하다는 것을 알았다. \par
 “EM은 충분한 시간만 주어진다면 수학을 ‘완성’시킬 수 있을 것이다. EM의 작업이 끝나게 된다면, 그것은 수학에서 더 이상 발견할 것이 없다는 뜻이다. 그때가 바로 수학이 완성되는 시점이다.” \par
 아버지는 그 어떤 때보다도 열의에 차 있었다. 에너지로 가득하다. 이런 아버지를 보는 것은 이번이 두 번째였다. \par
 “입력 완료까지 30분 남았다. 알맞은 시간에 왔구나. 연구동으로 내려가자.”
 \\ \par
 
 색색깔의 불빛이 계기판에 가득했다. 맥은 거대한 EM의 본체를 멍하니 지켜보았다. \par
 “10- 9- 8- ..... -1 , 가동.” \par
 환호성이 울려퍼졌다.
 \\ \par 
\chapter{?}

 테리사는 당황했다. 그녀의 눈앞에 처음 보는 메시지가 떠올라 있었다.
 \\ \par
 \textit{“COMPLETE."}
 \\ \par
 연구소가 설립된 지 올해로 꼭 100년이 된다. 그동안 EM은 수많은 메시지를 출력해왔지만, 이번 것은 처음이었다. EM은 수학자라는 직업을 없애버렸다. 훨씬 더 빠른 속도로, 많은 성질들을 발견하고 증명했으므로. 이로 인해 많은 논란이 있었지만 현재에는 종료된 상태였다.
어쨌건, 그녀가 당황한 이유는 EM이 작동을 중지했기 때문이었다. 그녀는 전력을 검사했고, 장비 오류는 없는지 점검했다. 아무런 문제도 없었다. 그녀는 밖으로 나와 밤하늘을 바라보았다. 
\\ \par
 하늘의 별들이 점점 사라져가고 있었다. 
\chapter*{해설을 겸한 후기}

수학에 관련한 소설을 써보고 많은 것을 느낄 수 있었다.
평소에 수학과 글쓰기를 좋아하는 편이었기에 둘을 결합하는 시도를 한번 해보고 싶었다. 좋은 기회가 되었다고 생각한다. 마지막의 별들이 사라져간다는 묘사는, 인류가 창조된 목적이 ‘수학’을 채우는 것이었다는 내 나름의 반전을 담은 것이다. 파피루스는 수학의 경계를 제공하기 위해 신이 보낸 것이었다. 수학이라는 무궁무진한 분야를 완전히 개척할 수 있다면 어떨까 하는 생각을 해본 것이 이 소설의 아이디어였다. 실제로는 괴델의 불완전성 정리에 의해 이러한 공리계는 없다는 것이 증명되었다. 상상하며 이야기를 만들어내는 글쓰기를 통해 수학에 대한 시각을 넓힐 수 있었다.
\end{document}