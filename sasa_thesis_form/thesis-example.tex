\documentclass{thesis-SJ}

\title[English title]{논문제목}
\author[洪吉東]{홍길동}{Hong, Gil Dong} 
\studentnumber{3311}

\advisor{권현우}{Hyunwoo Kwon}
\teacher{이예찬}{Yechan Lee}
\referee[1]{이순신}
\referee[2]{권율}

\date{2018년 11월 21일}{November 21st, 2018}

%\title[Test]{논문제목}
%\author{Hong, Gil Dong} 

\begin{document}
	\EnglishAbstract
	
	Lorem Ipsum is simply dummy text of the printing and typesetting industry. Lorem Ipsum has been the industry's standard dummy text ever since the 1500s, when an unknown printer took a galley of type and scrambled it to make a type specimen book. It has survived not only five centuries, but also the leap into electronic typesetting, remaining essentially unchanged. It was popularised in the 1960s with the release of Letraset sheets containing Lorem Ipsum passages, and more recently with desktop publishing software like Aldus PageMaker including versions of Lorem Ipsum.
	
	\EnglishKeywords{English keywords}
	
	\KoreanAbstract
	
	하수(河水)는 두 산 틈에서 나와 돌과 부딪쳐 싸우며, 그 놀란 파도와 성난 물머리와 우는 여울과 노한 물결과 슬픈 곡조와 원망하는 소리가 굽이쳐 돌면서, 우는 듯, 소리치는 듯, 바쁘게 호령하는 듯, 항상 장성을 깨뜨릴 형세가 있어, 전차(戰車) 만승(萬乘)과 전기(戰騎) 만대(萬隊)나 전포(戰砲) 만가(萬架)와 전고(戰鼓) 만좌(滿座)로써는 그 무너뜨리고 내뿜는 소리를 족히 형용할 수 없을 것이다. 모래 위에 큰 돌은 홀연히 떨어져 섰고, 강 언덕에 버드나무는 어둡고 컴컴하여 물지킴과 하수 귀신이 다투어 나와서 사람을 놀리는 듯한데, 좌우의 교리(蛟 )가 붙들려고 애쓰는 듯싶었다.
	
	\KoreanKeywords{Korean keywords}
	
	\tableofcontents
	
	\mainpartstart
	
	\chapter{Introduction} 
	
	Lorem Ipsum is simply dummy text of the printing and typesetting industry. Lorem Ipsum has been the industry's standard dummy text ever since the 1500s, when an unknown printer took a galley of type and scrambled it to make a type specimen book. It has survived not only five centuries, but also the leap into electronic typesetting, remaining essentially unchanged. It was popularised in the 1960s with the release of Letraset sheets containing Lorem Ipsum passages, and more recently with desktop publishing software like Aldus PageMaker including versions of Lorem Ipsum.
	
	\begin{table}
		\caption{Test}
		Write table in this way. 
	\end{table}
	
	
	\begin{figure}
		Write figure in this way. 
		\caption{Test}
	\end{figure}


\end{document}